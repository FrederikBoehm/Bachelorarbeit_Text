\chapter{Introduction}\label{ch:intro}

% You may have read about similar things in \cite{Goodliffe2007}.
% You can also write footnotes.\footnote{Footnotes will be positioned automatically.}
% \blindtext

% \blindtext

% \section{This is an Important Section}
% \blindtext

% \subsection{And an even more important subsection}
% \blindtext
\section{Motivation}
Business reports are an important source of information to make investment decisions.
Containing financial data like the balance sheet and the income statement or risk estimations and managment decisions, the investor can gain an insight into the company's business.
These reports are released periodically, usually every quarter or year, and lead to a revaluation of the company on the stock market:
Investors read the disclosed information and decide, whether they should continue to hold the stock, buy additional shares or whether they should sell them.
They can base their decision on quantitative data from the financial disclosure of the annual report or they can use the qualitative information contained in the text of the report.
While the quantitative data can be further processed immediately, this is not possible with the qualitative information from the annual report's text.
The text has to be read attentively first to be able to interpret it meaningful, which is related with a considerable amount of work.
Yet, the text of an annual report contains information about future development, like possible risks or new business areas.
Therefore it should be used to the same extent for investment decisions.

Since the reading and understanding of the reports is time consuming, the question arises whether this task can be supported or even completely accomplished by software.
The approach to this problem is to regard it as a text classification problem, where the text of the business report is classified as either \textit{positive} or \textit{negative}.
Recent developments in the field of natural language processing led to language models like \acs{BERT}, which achieves new levels of textual comprehension and even outperform humans in certain tasks \cite[p. 7]{Devlin2018}.
However these models are technically limited to short texts, therefore this thesis also examines a way to use these models for texts in document length.
The thesis is structured as follows:
Chapter \ref{ch:data} describes the data, that was used for the research and how it was processed, chapter \ref{ch:method} describes how \acs{BERT} was applied to the long texts and which classifiers were used.
Chapter \ref{ch:experiments} then presents the classification attempts that were made, chapter \ref{ch:outlook} and \ref{ch:summary} finally give an outlook and a summary of the research.
\section{State of research}
\textbf{Impact of business reports on stock prices} \\
There are several studies about the impact of business reports on the stock price.
\cite{Feldman2010} analyses the impact of tone changes in the management discussion and analysis (MD\&A) section of annual reports on the stock market \cite[p. 918]{Feldman2010}.
This tone change is measured by calculating the difference of the number of positive (negative) words to the average of positive (negative) words of annual reports in the last 400 days \cite[p. 927]{Feldman2010}.
The study finds that the tone change affects portfolio returns in a two day time window around the report filing date \cite[pp. 935-936]{Feldman2010}.

\cite{Azimi2019} examine how the sentiment in annual reports affects the company's stock price.
They use an automated way to determine the sentiment of the annual reports:
First, they manually label 8000 sentences as \textit{neutral}, \textit{positive} or \textit{negative} and create embeddings for the words in the sentences.
With these labeled sequences a \ac{LSTM} is trained, which can then be used to determine the sentiments of all other sentences in the reports \cite[p. 12]{Azimi2019}.
The sentiment of each report is then calculated by dividing the number of positive or negative sentences by the total number of sentences in a report \cite[p. 15]{Azimi2019}.
They find that the sentiment leads to a significant market reaction within three days after the annual report has been released, with a stronger reaction in case of negative sentiment compared to positive sentiment \cite[pp. 17-18]{Azimi2019}.

\textbf{Prediction of stock prices with machine learning} \\
\cite{Heo2016} predict a rise or a fall of the stock price in a one month period after the disclosure of the annual report.
They use ratios, calculated from the financial information of the report, like earnings per share, book-value per share or net profit growth rate as features to train a \ac{SVM} \cite[p. 61]{Heo2016}.
For that one month period they gain a prediction accuracy of up to 57.1\% \cite[p. 63]{Heo2016}.

\cite{Milosevic2016} follows a long term approach as they try to predict whether the stock price increases by more than 10\% or not in a one year period \cite[pp. 4-5]{Milosevic2016}.
They run two experiments, one with 28 financial indicators and one with 11 indicators, like market capitalization, dividend yield, net revenue growth rate or price to earnings ratio \cite[pp. 3-4]{Milosevic2016}.
They trained different classifiers including Decision Tree, \ac{SVM}, Random Forest, Logistic Regression and Naive Bayes \cite[p. 5]{Moukalled2019}.
For the experiment with 28 indicators they achieve a F-score of 75.1\% in case of the Random Forest and for the experiment with 11 indicators they achieve a F-score of 76.5\% on the same classifier \cite[p. 7]{Milosevic2016}.

\cite{Moukalled2019} try to predict whether the closing price of one day will be higher or lower than the closing price of the previous day.
This prediction is done for the stocks of four companies, which are Apple, Google, Amazon and Facebook by using news sentiments and historical stock prices.
The sentiments are determined by counting the number of positive and negative words in the news which leads to a division in neutral, positive and negative news \cite[p. 5]{Moukalled2019}.
7 other features were computed with the historical stock prices:
This includes the average of the stock price in the first trading hours and a trend indicator given by the slope of a linear model fitted to the stock price in this time frame \cite[p. 5]{Moukalled2019}.
They trained a \ac{RNN}, a feed forward neural network, a \ac{SVM} and support vector regression, giving directional accuracies between 75\% to 83\% in the case of the \ac{SVM} \cite[p. 8]{Moukalled2019}.

\textbf{Training of \acs{BERT} on annual reports} \\
Similar to this thesis, \cite{DeSola2019} also pre-trained \acs{BERT} on annual reports.
They trained three models with different results:
For FinBERT Prime, which they trained from scratch with business reports from 2017, 2018 and 2019, they gained a masked LM accuracy of 80.17\% and a next sentence accuracy of 98.5\% \cite[pp. 6-7]{DeSola2019}.
The second trained model was FinBERT Pre2K, which they trained on business reports from 1998 and 1999 yielding a masked LM accuracy of 77.2\% and a next sentence accuracy of 91.88\% \cite[pp. 6-7]{DeSola2019}.
Finally they also trained FinBERT Combo on top of the original BERT checkpoint, which gave a masked LM accuracy of 77.2\% and a next sentence accuracy of 90.63\% \cite[pp. 6-7]{DeSola2019}.
Their Finding is, that the language of business report changed within the 20 years, because the accuracies of FinBERT Prime were significantly lower than the accuracies of FinBERT Pre2K \cite[pp. 7-8]{DeSola2019}.

